\documentclass[DIN, pagenumber=false, fontsize=11pt, parskip=half]{scrartcl}

\usepackage{ngerman}
\usepackage[utf8]{inputenc}
\usepackage[T1]{fontenc}
\usepackage{textcomp}
\usepackage{enumitem}
\usepackage{amsmath}
\usepackage{amsfonts}
\usepackage{amssymb}
\usepackage{amsthm}
\setlength{\parindent}{0em}

% set section in CM
\setkomafont{section}{\normalfont\bfseries\Large}

\newcommand{\mytitle}[1]{{\noindent\Large\textbf{#1}}}
\newcommand{\mysection}[1]{\textbf{\section*{#1}}}
\newcommand{\mysubsection}[2]{\romannumeral #1) #2}


%===================================
\begin{document}

\noindent\textbf{Numerical Approximation of PDEs by Finite Differences and Finite Volumes} \hfill \textbf{Universität Hamburg}\\
SoSe 2023 \hfill Yahya Saleh\\

\mytitle{Exercise sheet 2 \hfill 05. May 2023}

%===================================
\mysection{Exercise 1}
Consider the ODE 
\begin{align*}
	\frac{d}{dt}x(t) &= \lambda x(t) \\
	x(t_0) &= x_0,
\end{align*}
for some $\lambda \in \mathbb{C}$.
Set $t_n = t_0 + n \Delta t$ for some $\Delta t > 0, n = 1, 2, \dots.$. Consider the following methods to propagate from $t_n$ to $t_{n+1}$:
\begin{itemize}
	\item Euler method: $$x_{n+1} = x_n + \Delta t \lambda x_n$$ \\
	\item Implicit Euler method:
	$$
	x_{n+1} = x_n + \Delta t  \lambda x_{n+1}
	$$
\end{itemize}
Determine the regions of stability for both methods. Assume that $\lambda$ lies in the left half-plane, i.e., it has a negative real part. Under what conditions on $\Delta t$ is the implicit Euler method stable? Assume that $\lambda$ is purely imaginary, i.e., it has a zero real part. Is Euler's method a good choice for solving the Cauchy problem defined above?
\mysection{Exercise 2}
Consider a $p$-th order accurate summation by part operator given by a grid $\underline{x}$, a first derivative operator $D$ and a mass matrix $M$.  
\begin{itemize}
	\item Assume that $M$ is diagonal and show that $\underline{1}^T M \underline{x}^{2q} = \frac{1}{2q+1} (x_{\text{max}}^{2q+1} - x^{2q+1}_{\text{min}})$ for $q\in \{1, \dots, p-1\}$.

	\textbf{Remark:} This shows that $M$ induces a quadrature rule that integrates monomials of even degrees $\leq 2p-1$ exactly. It, hence, completes the proof of theorem 1.15 in the lecture notes.
	
	\item Assuming that $M$ is not necessary diagonal, show that $\underline{1}^T M$ still yields a quadrature rule that is at least $(p-1)$-th order accurate.
\end{itemize}

\mysection{Exercise 3}
Consider the problem
	\begin{align*}
		\partial_t u(t,x) + a \partial_x u(t,x) &= 0 \qquad \text{ in } (0,T) \times [0,2] \\
		u(0,x) & = \sin(\pi x) \qquad \text{for } x \in [-1,1] \\
		u(t,0) & = -\sin(\pi t) \qquad \text{for } t \in [0,T] \\
	\end{align*}
for some $a=1/2$. 
Solve this equation using the SBP operator from example 1.10 and a third or fourth order Runge-Kuta method. Conduct convergence studies where you consider as a metric the $L^2$ error, i.e.,
$$
\|u - u_{\text{approximate}}\|_{L^2} = \biggl(\int_0 ^2 \bigl(u(x) - u_{\text{approximate}}(x)\bigr)^2 \ dx\biggr)^{1/2}.
 $$
 In the convergence studies differentiate between the transient region $\text{TR} = \{x \in [0,2] \ | \ x>at\}$ and the asymptotic region $\text{AR} = [0,2] \setminus \text{TR}$. Repeat the convergence study for different values of $a>0$ and a higher-order SBP operator. 
 
 \textbf{Hint:} To compute the error remember that the mass matrix $M$ induces a quadrature rule. You can impose the boundary condition either weakly or strongly. An example of a weak imposition is given in the lecture notes.
\mysection{Exercise 4}
Consider the system of equations
\begin{align*}
	\partial_t u(t,x) +  \partial_x v(t,x) &= 0 \qquad \text{ in } (0,T) \times (0,1) \\
	\partial_t v(t,x) -  \partial_x u(t,x) &= 0 \qquad \text{ in } (0,T) \times (0,1) \\
	u(0,x) & = \sin(2\pi x) \qquad \text{for } x \in [0,1] \\
	v(0,x) & = -\sin(2\pi x) \qquad \text{for } x \in [0,1] \\
	u(t,1) & = v(t,0) \qquad \text{for } t \in [0,T] \\
	v(t,1) & = u(t,1) \qquad \text{for } t \in [0,T] \\
	\end{align*}
 Obtain an energy estimate for this problem. Solve the problem using an SBP discretization in space and a third- or fourth-order Runge-Kuta method. Impose the boundary conditions weakly.
\end{document}